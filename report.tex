\documentclass[logo=images/logo.png]{tehranReport}
\usepackage[inline]{enumitem}
\usepackage{amsmath}
\usepackage{amssymb}
\usepackage{mathtools}
\usepackage[dvipsnames,table]{xcolor}
\usepackage{listings}
\usepackage{float}
\usepackage{geometry}
\usepackage{hyperref}
\usepackage{xepersian}
\usepackage{footnote}
\usepackage{pifont}

\settextfont{XB Niloofar}
\lstset{
    language=Python,
    columns=flexible,
    basicstyle=\ttfamily,
    numbers=left,
    numberstyle=\footnotesize\color{brown},
    keywordstyle=\bfseries\color{blue},
    keywordstyle={[2]\color{red}},
    keywordstyle={[3]\color{green}},
    commentstyle=\color{gray},
    stringstyle=\color{brown},
    captionpos=b,
    breaklines=true,
    breakatwhitespace=true,
    tabsize=4
}

\title{تمرین شماره‌ی ۱}
\author{علی رنجبر}
\authorPosition{دانشجوی کارشناسی مهندسی برق~-~مخابرات}
\university{دانشگاه تهران}
\college{پردیس دانشکده‌های فنی\\دانشکدهٔ مهندسی برق و کامپیوتر}
\studentNumber{810194321}
\course{آزمایشگاه آنتن (بهار ۹۸)}
\supervisor{دکتر کریم محمدپور اقدم}

\tolerance=5000
\renewcommand{\thesection}{\arabic{section}}
\renewcommand{\thesubsection}{\arabic{subsection}}

\newcommand{\cmark}{\textcolor{green!80!black}{\ding{51}}}
\newcommand{\xmark}{\textcolor{red}{\ding{55}}}

\makesavenoteenv{tabular}

\begin{document}
    \maketitlepage

    \section*{مقایسه تجهیزات اندازه‌گیری}
    

    \subsection*{مولدهای سیگنال\LTRfootnote{Signal Generators}}
	
	\begin{center}
		\begin{latin}
			\begin{tabular}{|c || c c|}
				\cline{2-3}
				\multicolumn{1}{c}{}&\multicolumn{1}{|c}{\textbf{E8257D-Agilent}}&\multicolumn{1}{c|}{\textbf{E8267D-Aginlent}}\\
				\hline\hline
			    Model & Analog Signal Generator & Vector Signal Generator\\
			    \hline
			    Step attenuator & optional & 0 to 115 dB in 5 dB steps \\ 
			    \hline
			    reference oscillator & 10 MHz & 10 MHz\\
			    \hline
			    Interfaces & \multicolumn{2}{c|}{RS–232, GPIB, and 10Base–T }\\
			    \hline
			    \multicolumn{3}{|l|}{\cellcolor[HTML]{F0F0F0}\textbf{Frequency}}\\
				\hline
				Range & 250 KHz to 67 GHz & 250 KHz to 44 GHz\\
				\hline
				Resolution & 0.001‌ Hz& 0.001 Hz\\
				\hline
				\multicolumn{3}{|l|}{\cellcolor[HTML]{F0F0F0}\textbf{Output}}\\
				\hline
				Minimum output power(dBm) & -20 (-135\footnote{With Option 1E1 step attenuator}) & -130\\
				\hline
				Maximum output power(dBm) & +28 & +21\\
				\hline
			\end{tabular}
		\end{latin}
	\end{center}
	
	\subsection*{تحلیل‌گر طیف\LTRfootnote{Spectrum Analyzers}}
	
	\begin{center}
		\begin{latin}
			\begin{tabular}{|c || c c|}
				\cline{2-3}
				\multicolumn{1}{c}{}&\multicolumn{1}{|c}{\textbf{E4448A-Agilent}}&\multicolumn{1}{c|}{\textbf{E4407B-Aginlent}}\\
				\hline\hline
				Families & PSA & ESA-E\\
				\hline
				Frequency Range(DC coupled) & 3 Hz to 50 GHz (325 GHz\footnote{With external mixers}) & 100 Hz to 26.5 GHz\\
				\hline
				Amplitude accuracy & 0.19 dB & 0.4 dB\\ 
				\hline
				Analysis bandwidth & 10(Standard), 40, 80 MHz & 10 MHz\\
				\hline
				Standard Attenuator Range & 70 dB(in 2 dB steps) & 60 dB(in 10 dB steps)\\
				\hline
			\end{tabular}
		\end{latin}
	\end{center}
\end{document}
